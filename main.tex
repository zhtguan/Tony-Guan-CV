%
%Academic CV LaTeX Template
% Author: Dubasi Pavan Kumar
% LinkedIn: https://www.linkedin.com/in/im-pavankumar/
% License: MIT
%
% For errors, suggestions, or improvements, please contact:
% Email: pavankumard.pg19.ma@nitp.ac.in
%



\documentclass[a4paper,11pt]{article}

% Package imports
\usepackage{latexsym}
\usepackage{xcolor}
\usepackage{float}
\usepackage{ragged2e}
\usepackage[empty]{fullpage}
\usepackage{wrapfig}
\usepackage{lipsum}
\usepackage{tabularx}
\usepackage{titlesec}
\usepackage{geometry}
\usepackage{marvosym}
\usepackage{verbatim}
\usepackage{enumitem}
\usepackage{fancyhdr}
\usepackage{multicol}
\usepackage{graphicx}
\usepackage{cfr-lm}
\usepackage[T1]{fontenc}
\usepackage{fontawesome5}
% \usepackage{xeCJK}
% Color definitions
\definecolor{darkblue}{RGB}{0,0,139}

% Page layout
\setlength{\multicolsep}{0pt} 
\pagestyle{fancy}
\fancyhf{} % clear all header and footer fields
\fancyfoot{}
\renewcommand{\headrulewidth}{0pt}
\renewcommand{\footrulewidth}{0pt}
\geometry{left=1.4cm, top=0.8cm, right=1.2cm, bottom=1cm}
\setlength{\footskip}{5pt} % Addressing fancyhdr warning

% Hyperlink setup (moved after fancyhdr to address warning)
\usepackage[hidelinks]{hyperref}
\hypersetup{
    colorlinks=true,
    linkcolor=darkblue,
    filecolor=darkblue,
    urlcolor=darkblue,
}

% Custom box settings
\usepackage[most]{tcolorbox}
\tcbset{
    frame code={},
    center title,
    left=0pt,
    right=0pt,
    top=0pt,
    bottom=0pt,
    colback=gray!20,
    colframe=white,
    width=\dimexpr\textwidth\relax,
    enlarge left by=-2mm,
    boxsep=4pt,
    arc=0pt,outer arc=0pt,
}

% URL style
\urlstyle{same}

% Text alignment
\raggedright
\setlength{\tabcolsep}{0in}

% Section formatting
\titleformat{\section}{
  \vspace{-4pt}\scshape\raggedright\large
}{}{0em}{}[\color{black}\titlerule \vspace{-7pt}]

% Custom commands
\newcommand{\resumeItem}[2]{
  \item{
    \textbf{#1}{\hspace{0.5mm}#2 \vspace{-0.5mm}}
  }
}

\newcommand{\resumePOR}[3]{
\vspace{0.5mm}\item
    \begin{tabular*}{0.97\textwidth}[t]{l@{\extracolsep{\fill}}r}
        \textbf{#1}\hspace{0.3mm}#2 & \textit{\small{#3}} 
    \end{tabular*}
    \vspace{-2mm}
}

\newcommand{\resumeSubheading}[4]{
\vspace{0.5mm}\item
    \begin{tabular*}{0.98\textwidth}[t]{l@{\extracolsep{\fill}}r}
        \textbf{#1} & \textit{\footnotesize{#4}} \\
        \textit{\footnotesize{#3}} &  \footnotesize{#2}\\
    \end{tabular*}
    \vspace{-2.4mm}
}

\newcommand{\resumeProject}[4]{
\vspace{0.5mm}\item
    \begin{tabular*}{0.98\textwidth}[t]{l@{\extracolsep{\fill}}r}
        {\textbf{#1}} & \textit{\footnotesize{#3}} \\
        \footnotesize{\textit{#2}} & \footnotesize{#4}
    \end{tabular*}
    \vspace{-2.4mm}
}


\newcommand{\resumeSubItem}[2]{\resumeItem{#1}{#2}\vspace{-4pt}}

\renewcommand{\labelitemi}{$\vcenter{\hbox{\tiny$\bullet$}}$}
\renewcommand{\labelitemii}{$\vcenter{\hbox{\tiny$\circ$}}$}

\newcommand{\resumeSubHeadingListStart}{\begin{itemize}[leftmargin=*,labelsep=1mm]}
\newcommand{\resumeHeadingSkillStart}{\begin{itemize}[leftmargin=*,itemsep=1.7mm, rightmargin=2ex]}
\newcommand{\resumeItemListStart}{\begin{itemize}[leftmargin=*,labelsep=1mm,itemsep=0.5mm]}

\newcommand{\resumeSubHeadingListEnd}{\end{itemize}\vspace{2mm}}
\newcommand{\resumeHeadingSkillEnd}{\end{itemize}\vspace{-2mm}}
\newcommand{\resumeItemListEnd}{\end{itemize}\vspace{-2mm}}
\newcommand{\cvsection}[1]{%
\vspace{2mm}
\begin{tcolorbox}
    \textbf{\large #1}
\end{tcolorbox}
    \vspace{-4mm}
}

\newcolumntype{L}{>{\raggedright\arraybackslash}X}%
\newcolumntype{R}{>{\raggedleft\arraybackslash}X}%
\newcolumntype{C}{>{\centering\arraybackslash}X}%

% Commands for icon sizing and positioning
\newcommand{\socialicon}[1]{\raisebox{-0.05em}{\resizebox{!}{1em}{#1}}}
\newcommand{\ieeeicon}[1]{\raisebox{-0.3em}{\resizebox{!}{1.3em}{#1}}}

% Font options
\newcommand{\headerfonti}{\fontfamily{phv}\selectfont} % Helvetica-like (similar to Arial/Calibri)
\newcommand{\headerfontii}{\fontfamily{ptm}\selectfont} % Times-like (similar to Times New Roman)
\newcommand{\headerfontiii}{\fontfamily{ppl}\selectfont} % Palatino (elegant serif)
\newcommand{\headerfontiv}{\fontfamily{pbk}\selectfont} % Bookman (readable serif)
\newcommand{\headerfontv}{\fontfamily{pag}\selectfont} % Avant Garde-like (similar to Trebuchet MS)
\newcommand{\headerfontvi}{\fontfamily{cmss}\selectfont} % Computer Modern Sans Serif
\newcommand{\headerfontvii}{\fontfamily{qhv}\selectfont} % Quasi-Helvetica (another Arial/Calibri alternative)
\newcommand{\headerfontviii}{\fontfamily{qpl}\selectfont} % Quasi-Palatino (another elegant serif option)
\newcommand{\headerfontix}{\fontfamily{qtm}\selectfont} % Quasi-Times (another Times New Roman alternative)
\newcommand{\headerfontx}{\fontfamily{bch}\selectfont} % Charter (clean serif font)

\begin{document}
\headerfontiii

% Header
\begin{center}
    {\Huge\textbf{Zhongtao(Tony) Guan}}
\end{center}
\vspace{-4mm}
\begin{center}
    \small{
    +1-617-784-6900 | 
    \href{mailto:zhtguan@mit.edu}{zhtguan@mit.edu}|
    \socialicon{\faLinkedin} \href{https://www.linkedin.com/in/zhongtao-guan-220766287/}{Linkedin} | 
    \socialicon{\faGithub} \href{https://github.com/DoZz-M}{Github}
    }
\end{center}
\vspace{-4mm}

\section{\textbf{Education and Research Experience}}
\vspace{-0.4mm}
\resumeSubHeadingListStart

\resumeSubheading
{\href{https://shanghaitech.edu.cn}{ShanghaiTech University}}{Shanghai, China}
{Bachelor of Engineering, Electronic Information Engineering}{Sep. 2021 - Present}
\resumeItemListStart
\item GPA: 3.80/4.00; Ranking 3/56
\item Core courses:  Introduction to Control, Signals and Systems, Electromagnetic, Power Electronics
% \item  Honors: Outstanding Teaching assistant( Head TA of Electric Circuit); Excellent Student
\item  Scholarship: Undergrad. National Exchange Scholarship; International Conference Scholarship
\resumeItemListEnd

\resumeSubheading
{\href{https://mit.edu}{Massachusetts Institute of Technology}}{Cambridge, Massachusetts, U.S.}
{Special Student Program in EECS}{Feb. 2024 - May.2024}
\resumeItemListStart
\item GPA: 5.00/5.00
\item Core courses:  Underactuated Robotics, Nonlinear Control
\resumeItemListEnd

\resumeSubheading
{\href{https://mit.edu}{Massachusetts Institute of Technology}}{Cambridge, Massachusetts, U.S.}
{Undergraduate Visiting Student in EECS}{July. 2024 - Present}
\resumeItemListStart
\item Advisor: Kevin Chen
\resumeItemListEnd
\vspace{-2mm}

\section{\textbf{Publications} \hspace*{\fill} \textcolor{darkblue}{\scriptsize C=Conference, J=Journal, S=In Submission, +=equal contribution}}
\vspace{0.2mm}
\small{
\begin{enumerate}[leftmargin=*, labelsep=0.5em, align=left, widest={[\textbf{S.1}]}, itemindent=0em, label={\textbf{[\arabic*]}]}]

\item[\textbf{[S.1]}] Yi-Hsuan Hsiao$^+$, Songnan Bai$^+$, \textbf{Zhongtao Guan$^+$}, et al. \textbf{Hybrid locomotion at the insect scale – combined flying and jumping for enhanced efficiency and efficacy}. Manuscript submitted for publication in \textit{ Nature Machine Intelligence}.\label{S.1}

\item[\textbf{[C.1]}] \textbf{Zhongtao Guan}, et al. \href{https://github.com/horychen/CuryLegWebotSimulation}{\textbf{Preliminary Result of Cury: A Backdrivable Leg Design using Linear Actuators}}. In \textit{IEEE/RSJ International Conference on Intelligent Robots and Systems(IROS), 2024}. \label{C.1}

\item[\textbf{[C.2]}] \textbf{Zhongtao Guan}, et al. \href{https://ieeexplore.ieee.org/abstract/document/10253208}{\textbf{Accurate Single-Ended Fault Location for Cable-OHL Hybrid Transmission Lines}}. In \textit{Power and Energy Society General Meeting (PESGM), 2023}. \label{C.2}

\item[\textbf{[C.3]}] Jiayu Yang, Yu Liu, Kang Yue, \textbf{Zhongtao Guan}, et al. \href{https://ieeexplore.ieee.org/abstract/document/10253748}{\textbf{Closed-Form Solutions of Mutual Inductance and Load for LCC-S Wireless Power Transfer Systems}}. In \textit{3rd IEEE International Conference on Industrial Electronics for Sustainable Energy Systems, 2023}. \label{C.3}

\item[\textbf{[C.4]}] Mengzhao Duan, Yu Liu, Ze Liu, Xinchen Zou and \textbf{Zhongtao Guan}. \href{https://ieeexplore.ieee.org/abstract/document/10252886}{\textbf{A Group of Single-Ended Time-Domain Line Fault Location Methods Using Breaker Operation Information}}. In \textit{IEEE Power and Energy Society General Meeting (PESGM), 2023}.\label{C.4}
\end{enumerate}
}

\section{\textbf{Projects}}
\vspace{-0.4mm}
\resumeSubHeadingListStart

\resumeProject
  {Implicit Regularization and Dynamic Gain in Nonlinear Control}
  {Advisor: Prof. Jiahao Chen}
  {Sep. 2023- Jan. 2024}
    {{}}
\resumeItemListStart
 \item Place Holder
 \item Place Holder
 \item Place Holder
 \item Place Holder
 \item Place Holder
 \item Place Holder
 
\resumeItemListEnd

\resumeProject
  {\textbf{Sensor Autonomy for Insect-Scale Robots}}
  {Advisor: Prof. Kevin Chen}
  {July. 2024- Present}
    {{}}
\resumeItemListStart
 \item Place Holder
 \item Place Holder
 \item Place Holder
 \item Place Holder
 \item Place Holder
 
\resumeItemListEnd

\resumeProject
  {\textbf{Hybrid Locomotion at Insect Scale }}
  {Advisor: Prof. Kevin Chen}
  {Jan. 2024- Sep. 2024}
    {{}}
\resumeItemListStart
  % \item [Abilities] *Abilities, including, overcoming tall obstacles, performing somersaults between consecutive jumps, 
  % \item [Terrian] traversing challenging surfaces such as soil, slippery glass, rotating surfaces, and quadrotor platform.
  % \item [Resilience] \textcolor{red}{Properties} collision resilience under large impact.

  \item Presented a sub-gram flapping-wing passive hopper at insect scale using soft actuator.
  \item Demonstrated capabilities in overcoming obstacles, navigating challenging terrains, and exhibiting high agility. 
    \item Trajectory optimization and online NLMPC are used for complex task such as fast dynamic between slopes.
  \item Contributed to controller design, experiments and data analysis. 
  \item This work is submitted to a journal: [S.1].
\resumeItemListEnd
% \newpage
\resumeProject
  {\textbf{A Backdrivable Leg Design Using Linear Actuators}}
  {Advisor: Prof. Jiahao Chen}
  {Aug. 2023 - Jan. 2024}
  {{}[\href{https://github.com/horychen/CuryLegWebotSimulation}{\textcolor{darkblue}{\faGithub}}]}
   % \item \textcolor{red}{*Designed the completed hardware pipeline, including four-bar mechanisms, AC motors, and its drive;} Utilized multi-objective optimization for parameter selection
% \item Designed  AC motors,drives, fundamental controller and parameters selection of four-bar linkages using Utilized multi-objective optimization(MOO).
  % \item Acted as the project leader; Responsible for the design, fabrication, and assembly of mechanical systems, as well as contributing to the design, testing, and control of motors and drive systems.
\resumeItemListStart
 \item Developed a backdrivable 2-DoF leg prototype for walking and jumping.
    \item Contributed to the design of electronic components, including a highly integrated AC motor drive.
    \item Reduced the number of joint encoders through optimized mechanical design and electrical integration.
    \item Built a simulation environment using the Webots simulator for closed-loop chain dynamics.
    \item Acted as the project leader; responsible for mechatronics design and simulation.
    \item This work has been accepted as a conference paper: [C.1].
\resumeItemListEnd

\resumeProject
  {Fault Location of Power Systems}
  {Advisor: Prof. Yu Liu}
  {Jun. 2022 - Jan. 2023}
  {{}}
\resumeItemListStart
  % \item HOW1 \& ROLE \& IMPACT(Paper1)
  %   1: 对输电线进行了更精确的建模,并且对Eriksson Method进行了拓展使其可以被部署
  %   2: 使用完全分布参数模型,解决了多解问题,在高/低阻故障下有很好的定位精度
  %   This work is accepted as an conference paper [C.2]
  % \item HOW2 \& ROLE \& IMPACT(Paper2) Introduced the breaker operation information to the single-ended time-domain fault location problem on AC transmission line, 暴打之前所有经典方法. This work is accepted as an conference paper [C.4]
\item  Proposed methods for fault location on hybrid or purely overhead line power system.
\item Utilized fully distributed line model for accurate locating, while modified Eriksson method for analytical method .
\item Introduced breaker operation information for fault location of pure overhead-line power system.
\item  Contributed to idea, methodology, experiments for [C.2]; proof reading and discussion for [C.4]. 
\item These works are accepted as conference papers [C.2], [C.4].
\resumeItemListEnd

\resumeProject
  {Design and Control of Inverter}
  {Advisor: Prof. Yu Liu}
  {Jan. 2023 - Aug. 2023}
  {{}}
\resumeItemListStart
  \item Proposed analytical solutions of mutual inductance and load resistance for the LCC-S WPT system, without communication from the secondary side.
  \item Designed and controlled a three-phase inverter for grid-connected photovoltaic systems.
  % \item Included knowledge of device selection, embedded system, SVPWM and PLL.
  \item This work is accepted as a conference paper: [C.3] %and National Undergraduate Electronic Design Contest.
\resumeItemListEnd

\resumeSubHeadingListEnd

\section{\textbf{Awards}}
\resumeProject
  {Outstanding Teaching Assistant}
  {ShanghaiTech University, school of information and technology}
  {2023}
  {{}}
\resumeItemListStart
  \item Acted as head TA  Electric Circuit.
  \item Coordinated the workload of TAs, Recorded the class, lectured discussion/review session, graded homework.
\resumeItemListEnd

\resumeProject
  {RoboMaster University Championship }
  {RoboMaster}
  {2022}
  {{}}
\resumeItemListStart
  \item Won 2nd Prize in Shanghai division, 3rd in national division 
  \item Acted as group leader; contributed to mechanical design.

\resumeItemListEnd

\resumeProject
  {National Undergraduate Electronic Design Contest}
  {Shanghai Municipal Education Commission }
  {2023}
  {{}}
\resumeItemListStart
  \item Won 2nd Prize in Shanghai division
  \item Acted as group leader; contributed to inverter design and control.
    \item Included knowledge of device selection, embedded system, SVPWM and PLL.
\resumeItemListEnd

\section{\textbf{Skills and Others }}
\vspace{-0.4mm}
 \resumeHeadingSkillStart
  \resumeSubItem{Programming Languages: }
    {Python, C/C++, Julia, Matlab}
    \resumeSubItem{Toolkit: }
    {Simulink, Altuim Designer, KiCAD, Solidworks, \LaTeX}
    \resumeSubItem{Teaching: }{Electric Circuit, Introduction to Control Project}
    \resumeSubItem{\textcolor{red}{Research}: }{}
 \resumeHeadingSkillEnd
 
\resumeSubHeadingListEnd

\section{\textbf{References}}
\vspace{-0.2mm}
\resumeHeadingSkillStart
\resumeSubItem{Research Supervisor: }{\href{smrl.mit.edu}{Kevin Chen}, Associate Professor Without Tenure, MIT, Contact: \href{mailto:yufengc@mit.edu}{Email}}
\resumeSubItem{Research Supervisor: }{\href{https://faculty.sist.shanghaitech.edu.cn/chenjh/}{Jiahao Chen}, Assistant Professor, ShanghaiTech, Contact:\href{mailto:chenjh2@shanghaitech.edu.cn}{Email}}
\resumeSubItem{Research Supervisor: }{\href{https://pspal.shanghaitech.edu.cn/people.html}{Yu Liu}, Associate Professor, ShanghaiTech, Contact: \href{mailto:liuyu@shanghaitech.edu.cn}{Email}}
 \resumeHeadingSkillEnd
\end{document}